\documentclass[french, 12pt]{article}
\usepackage[a4paper, top=2cm, bottom=2cm, left=2cm, right=2cm]{geometry}

\usepackage{fancyvrb}

\usepackage{lmodern}
\usepackage[T1]{fontenc}
\usepackage[french]{babel}

\usepackage[indent]{parskip}
\usepackage{multicol}
\usepackage{multirow}
\usepackage{amssymb}
\usepackage{calc}
\usepackage{enumitem}
\usepackage{array}
\usepackage{hyperref}
\usepackage{tabularx}
\usepackage{booktabs}
\usepackage{bytefield}
\usepackage{rotating}
\usepackage{register}
\usepackage[table]{xcolor}

\usepackage{todonotes}

\newcommand{\bitlabel}[2]{%
\bitbox[]{#1}{%
\raisebox{0pt}[4ex][0pt]{%
\turnbox{50}{\fontsize{7}{7}\selectfont#2}%
}%
}%
}

\definecolor{black}         {RGB}{12,  12,  12 }
\definecolor{red}           {RGB}{197, 15,  31 }
\definecolor{green}         {RGB}{19,  161, 14 }
\definecolor{yellow}        {RGB}{193, 156, 0  }
\definecolor{blue}          {RGB}{0,   55,  218}
\definecolor{magenta}       {RGB}{136, 25,  152}
\definecolor{cyan}          {RGB}{58,  150, 221}
\definecolor{white}         {RGB}{204, 204, 204}
\definecolor{bright_black}  {RGB}{118, 118, 118}
\definecolor{bright_red}    {RGB}{231, 72,  86 }
\definecolor{bright_green}  {RGB}{22,  198, 12 }
\definecolor{bright_yellow} {RGB}{249, 241, 165}
\definecolor{bright_blue}   {RGB}{59,  120, 255}
\definecolor{bright_magenta}{RGB}{180, 0,   158}
\definecolor{bright_cyan}   {RGB}{97,  214, 214}
\definecolor{bright_white}  {RGB}{242, 242, 242}

\setlength{\marginparwidth }{2cm}

\begin{document}

\section*{Syntaxe d'un fichier}
Les commentaires commencent par \texttt{;} et s'étendent jusqu'à la fin de la ligne.

Les sections disponibles sont les suivantes :
\begin{description}
      \item[\texttt{.text}] signale le début d'une section d'instructions assembleur.
      \item[\texttt{.data}] signale le début d'une section de déclarations de données.
\end{description}


Chaque instruction ou déclaration peut être localisée par un \emph{label} qui
sera converti en une adresse par l'assembleur. Un label respecte la \emph{regex} suivante :
\texttt{[a-z,A-Z,\_,0-9]$^{+}$}.

On localise une instruction/déclaration à l'aide de la syntaxe :
\begin{verbatim}
label:
      instruction ou déclaration
\end{verbatim}


On note que l'adressage de la mémoire se fait par mots de 32 bits (et pas moins !).
Pour faire appel à des fonctions ou des sections définies dans un autre fichier
on utilise la syntaxe suivante à n'importe quel endroit du fichier :
\verb|.include file|. Dans ce cas, le code du fichier référencé est inséré
à cette position.

Les dossiers utilisés pour la recherche des fichiers à insérer est configurable
via la ligne de commande.

\section*{Données}

\subsection*{Chaines de caractères}

Les chaines de caractères sont composées des caractères suivants :
\begin{itemize}
      \item Les caractères ASCII affichables (codes $32$ à $126$ inclus).
      \item Le caractère nul noté \verb|\0| (code $0$).
\end{itemize}
Les caractères \texttt{"} et les \texttt{\textbackslash} sont échappés :
\texttt{\textbackslash{}"} pour le guillemet et
\texttt{\textbackslash{}\textbackslash{}} pour la barre oblique inversée.
La \autoref{fig:char} expose comment les caractères sont codés sur des
mots de $32$ bits.

\begin{figure}[htp]
      \begin{center}
            \vspace{0.5cm}% Parce que la taille des truc penchés n'est pas hyper bien calculée
            \begin{bytefield}[bitwidth=0.03\textwidth]{32}
                  \bitlabel{6}{} &
                  \bitlabel{1}{Overline} &
                  \bitlabel{1}{Crossed} &
                  \bitlabel{1}{Hide} &
                  \bitlabel{1}{Blinking} &
                  \bitlabel{1}{Underline} &
                  \bitlabel{1}{Italic} &
                  \bitlabel{1}{Faint} &
                  \bitlabel{1}{Bold}
                  \\
                  \bitheader[endianness=big]{0,6-8,12,13,17-26,31} \\
                  \bitbox{6}{0} &
                  \bitbox{8}{Style} &
                  \bitbox{5}{BG} &
                  \bitbox{5}{FG} &
                  \bitbox{1}{0} &
                  \bitbox{7}{Code ASCII}
            \end{bytefield}
      \end{center}
      \caption{Encodage d'un caractère.}
      \label{fig:char}
\end{figure}

Avec :
\begin{itemize}
      \item Code ASCII : Le code ASCII du caractère représenté.
      \item FG : La couleur du texte comme décrit dans le \autoref{tbl:color}.
      \item BG : La couleur du derrière du texte comme décrit dans le \autoref{tbl:color}.
\end{itemize}

\newpage

\renewcommand{\arraystretch}{1.1}
\begin{table}[htp]
      \begin{center}
            \begin{tabular}{clp{1.5em}}
                  \toprule
                  Identifiant & Nom           &                            \\
                  \midrule
                  0           & Noir          & \cellcolor{black}          \\
                  1           & Rouge         & \cellcolor{red}            \\
                  2           & Vert          & \cellcolor{green}          \\
                  3           & Jaune         & \cellcolor{yellow}         \\
                  4           & Bleu          & \cellcolor{blue}           \\
                  5           & Magenta       & \cellcolor{magenta}        \\
                  6           & Cyan          & \cellcolor{cyan}           \\
                  7           & Blanc         & \cellcolor{white}          \\
                  8           & Noir Clair    & \cellcolor{bright_black}   \\
                  9           & Rouge Clair   & \cellcolor{bright_red}     \\
                  10          & Vert Clair    & \cellcolor{bright_green}   \\
                  11          & Jaune Clair   & \cellcolor{bright_yellow}  \\
                  12          & Bleu Clair    & \cellcolor{bright_blue}    \\
                  13          & Magenta Clair & \cellcolor{bright_magenta} \\
                  14          & Cyan Clair    & \cellcolor{bright_cyan}    \\
                  15          & Blanc Clair   & \cellcolor{bright_white}   \\
                  16          & Défaut        &                            \\
                  \bottomrule
            \end{tabular}
      \end{center}
      \caption{Couleurs disponibles et leurs identifiants.}
      \label{tbl:color}
\end{table}

Le texte est par défaut donné entre guillement : \texttt{"texte"}. Les
fonctions suivantes sont disponibles afin de spécifier les couleurs ainsi que
le style du texte :
\begin{itemize}
      \item \texttt{\#textcolor(}\textit{couleur}, \texttt{"}\textit{texte}\texttt{")} pour spécifier la couleur du texte.
      \item \texttt{\#backcolor(}\textit{couleur}, \texttt{"}\textit{texte}\texttt{")} pour spécifier la couleur du fond.
      \item \texttt{\#bold(}\texttt{"}\textit{texte}\texttt{")}
      \item \texttt{\#faint(}\texttt{"}\textit{texte}\texttt{")}
      \item \texttt{\#italic(}\texttt{"}\textit{texte}\texttt{")}
      \item \texttt{\#underline(}\texttt{"}\textit{texte}\texttt{")}
      \item \texttt{\#blinking(}\texttt{"}\textit{texte}\texttt{")}
      \item \texttt{\#hide(}\texttt{"}\textit{texte}\texttt{")}
      \item \texttt{\#crossed(}\texttt{"}\textit{texte}\texttt{")}
      \item \texttt{\#overlined(}\texttt{"}\textit{texte}\texttt{")}
      \item \texttt{\#default(}\texttt{"}\textit{texte}\texttt{")}
\end{itemize}

Le symbole \texttt{+} pourra être utilisé afin de concaténer du texte.
Par exemple :
\begin{center}
      \noindent\texttt{\#bold(\#textcolor(red, "Hello") + " " + \#italic(\#textcolor("green", "World")))}  \\
      \textdownarrow\\
      \noindent\texttt{\textbf{\color{red}{Hello}} \textit{\color{green}{World}}}
\end{center}

\subsection*{Entiers}

Les entiers peuvent être donnés dans plusieurs bases :
\begin{itemize}
      \item En base décimale, base par défaut.
      \item En base binaire, lorsque préfixés par \texttt{0b}.
      \item En base hexadécimale, lorsque préfixés par \texttt{0x}.
\end{itemize}

\subsection*{Déclaration des données}

On utilise les notations suivantes pour déclarer des données :
\begin{description}[format=\normalfont, leftmargin=!, labelwidth=\widthof{\textbf{\texttt{.string}} \textit{text}}]
      \item[\textbf{\texttt{.string}} \textit{text}] déclare une chaîne de caractère. La chaîne est eventuellement stylisée.
            Cette chaîne n'est pas automatiquement terminée par \verb|\0|.
      \item[\textbf{\texttt{.zstring}} \textit{text}] déclare une chaîne de caractère
            terminée par \verb|\0|. La chaîne est eventuellement stylisée sauf le zéro final.
      \item[\textbf{\texttt{.int}}] déclare un entier signé sur $32$ bits signé ou non.
\end{description}

\section*{Instructions}

\subsection*{Registres}

L'assembleur possède $31$ registres de travail : de \texttt{r0} à \texttt{r28},
\texttt{rout}, \texttt{sp}, \texttt{fp}. Les registres suivants ont un sens
particulier :

\begin{description}[leftmargin=!, labelwidth=\widthof{\bf \texttt{rout}}]
      \item[\texttt{r0}] n'est pas modifiable et a comme valeur $0$.
      \item[\texttt{r1}] n'est pas modifiable et a comme valeur $1$.
      \item[\texttt{rout}] est supposé être utilisé afin de stocker la valeur de retour des fonctions.
      \item[\texttt{sp}] a comme valeur la prochaine adresse libre du tas.
      \item[\texttt{fp}] a comme valeur l'adresse du tableau d'activation de la fonction en cours.
\end{description}

\todo[inline]{
      Le registre $31$ est réservé à l'assembleur/compilateur.
      Avertissement dans l'assembleur si utilisé.
      Nom : \texttt{rpriv}
}

\subsection*{Convention d'appel}

La pile est organisée de cette manière lors d'un appel de fonction :
\begin{Verbatim}[baselinestretch=0.2]
            |  ...  |
            |  ...  |  caller values
            |  ...  |
            |       |
            |       |  caller saved registers
            |       |
            |-------|
            |  ret  |  address to jump at after the function
      fp -> | oldfp |  address of the previous stack frame
            |-------|
            |       |
            |       |  callee saved registers
            |       |
            |  ...  |
            |  ...  |
            |  ...  |  callee values
      sp -> |       |
            |       |
            |       |
\end{Verbatim}


Les 9 premiers arguments d'une fonction sont donnés dans les registres \texttt{r20} à \texttt{r28}.
Les autres sont passés sur la pile de la droite vers la gauche. C'est la responsabilité
de l'appeleur de nettoyer les arguments sur la pile après l'appel.

La valeur de retour de la fonction est passé dans le registre \texttt{rout}. Dans le cas
d'une structure ou de donnée trop grande, le registre \texttt{rout} contient une adresse
mémoire vers ces données.

Les registres de \texttt{r15} à \texttt{r28} sont dits \emph{caller-saved} (volatile) :
une fonction est susceptible d'écraser la valeur de ces registres sans les restaurer.
Les registres de \texttt{r0} à \texttt{r14} sont dits \emph{callee-saved} (non-volatile) :
lors de l'appel à une fonction, cette dernière ne doit pas modifier ces registres.

\subsection*{Labels}

Les \emph{labels} ne peuvent être déclarés qu'une seule fois.

\subsection*{Format des Sauts}

Lors de saut les formats d'adresse suivants sont autorisés :
\begin{itemize}
      \item Saut à une adresse absolue. L'adresse est donnée sans signe dans
            n'importe quelle base. Exemple : \verb|jmp 12| saute à l'adresse $12$ du programme.
      \item Saut à une adresse relative. L'adresse est donnée avec un signe
            (\texttt{+} ou \texttt{-}) dans n'importe quelle base. Exemple :
            \verb|jmp +12| saute $12$ instruction après, \verb|jmp -12| saute
            $12$ instruction avant.
      \item Saut à un \emph{label}. L'adresse absolue est calculée par
            l'assembleur. Exemple : \verb|jmp lbl| saute au label \emph{lbl} du
            programme.
\end{itemize}

\subsection*{Instructions}

\renewcommand\tabularxcolumn[1]{m{#1}}
\renewcommand{\arraystretch}{1.5}

\subsubsection*{Instructions de Calculs Logiques}
\noindent
\begin{tabularx}{\textwidth}{cccc X}
      \toprule
      Instruction  & Destination & \multicolumn{2}{c}{Arguments} & Description                                             \\
      \midrule
      \texttt{and} & $r_1$       & $r_2$                         & $r_3$       & Calcule $r_2 \land r_3$ dans $r_1$.       \\
      \texttt{or}  & $r_1$       & $r_2$                         & $r_3$       & Calcule $r_2 \lor r_3$ dans $r_1$.        \\
      \texttt{nor} & $r_1$       & $r_2$                         & $r_3$       & Calcule $\neg (r_2 \lor r_3)$ dans $r_1$. \\
      \texttt{xor} & $r_1$       & $r_2$                         & $r_3$       & Calcule $r_2 \oplus r_3$ dans $r_1$.      \\
      \texttt{not} & $r_1$       & $r_2$                         &             & Calcule $\neg r_2$ dans $r_1$.            \\
      \bottomrule
\end{tabularx}

\subsubsection*{Instructions de Calculs Arithmétiques}

\noindent
\begin{tabularx}{\textwidth}{cccc X}
      \toprule
      Instruction  & Destination & \multicolumn{2}{c}{Arguments} & Description                                        \\
      \midrule
      \texttt{add} & $r_1$       & $r_2$                         & $r_3$       & Calcule $r_2 + r_3$ dans $r_1$.      \\
      \texttt{sub} & $r_1$       & $r_2$                         & $r_3$       & Calcule $r_2 - r_3$ dans $r_1$.      \\
      \texttt{mul} & $r_1$       & $r_2$                         & $r_3$       & Calcule $r_2 \times r_3$ dans $r_1$. \\
      \texttt{div} & $r_1$       & $r_2$                         & $r_3$       & Calcule $r_2 \div r_3$ dans $r_1$.   \\
      \texttt{neg} & $r_1$       & $r_2$                         &             & Calcule $-r_2$ dans $r_1$.           \\
      \texttt{inc} & $r_1$       & $r_2$                         &             & Calcule $r_2 + 1$ dans $r_1$.        \\
      \texttt{dec} & $r_1$       & $r_2$                         &             & Calcule $r_2 - 1$ dans $r_1$.        \\
      \bottomrule
\end{tabularx}

\subsubsection*{Instructions de Décalages}

Les décalages sont opérés modulo $32$ bits.

\noindent
\begin{tabularx}{\textwidth}{cccc X}
      \toprule
      Instruction  & Destination & \multicolumn{2}{c}{Arguments} & Description                                                                                         \\
      \midrule
      \texttt{asr} & $r_1$       & $r_2$                         & $r_3$       & Décale $r_2$ vers la droite de $r_3$ bits en dupliquant le bit de signe dans $r_1$.   \\
      \texttt{lsr} & $r_1$       & $r_2$                         & $r_3$       & Décale $r_2$ vers la droite de $r_3$ bits en ajoutant les $0$ nécessaires dans $r_1$. \\
      \texttt{lrl} & $r_1$       & $r_2$                         & $r_3$       & Décale $r_2$ vers la gauche de $r_3$ bits en ajoutant les $0$ nécessaires dans $r_1$. \\
      \bottomrule
\end{tabularx}

\subsubsection*{Instructions pour la manipulation du Tas}

\noindent
\begin{tabularx}{\textwidth}{ccc X}
      \toprule
      Instruction   & Destination & Argument & Description                         \\
      \midrule
      \texttt{push} &             & $r_1$    & Empile $r_1$ sur le tas.            \\
      \texttt{pop}  & $r_1$       &          & Dépile le sommet du tas dans $r_1$. \\
      \bottomrule
\end{tabularx}

\subsubsection*{Instructions pour manipuler la Mémoire}

\noindent
\begin{tabularx}{\textwidth}{cccc X}
      \toprule
      Instruction    & Destination & \multicolumn{2}{c}{Arguments} & Description                                                                                                                             \\
      \midrule
      \texttt{mov}   & $r_1$       & \multicolumn{2}{l}{$r_2$}     & Copie le registre $r_2$ dans $r_1$.                                                                                                     \\
      \texttt{store} & $r_1$       & \multicolumn{2}{l}{$r_2$}     & Copie le registre $r_2$ à l'adresse mémoire de $r_1$.                                                                                   \\
      \texttt{load}  & $r_1$       & \multicolumn{2}{l}{$r_2$}     & Copie la valeur de la mémoire à l'adresse $r_2$ dans $r_1$.                                                                             \\
      \texttt{loadi} & $r_1$       & \multicolumn{2}{l}{$i$}       & Copie $i$ (entier signé ou non sur $32$ bits) dans $r_1$.                                                                               \\
      \texttt{loadi} & $r_1$       & \multicolumn{2}{l}{$\$l$}     & Copie l'adresse associée à $l$ dans $r_1$.                                                                                              \\
      \texttt{loadi} & $r_1$       & $i$                           & $r_2$                                                       & Copie $i + r_2$ dans $r_1$ avec $i$ un entier signé ou non sur $32$ bits. \\
      \texttt{loadi} & $r_1$       & $\$l$                         & $r_2$                                                       & Copie l'adresse associée à $l$ plus $r_2$ dans $r_1$.                     \\
      \bottomrule
\end{tabularx}

\subsubsection*{Instructions de Branchement}

\noindent
\begin{tabularx}{\textwidth}{ccc X}
      \toprule
      Instruction             & Destination & Argument & Description                                                                                                                                 \\
      \midrule
      \texttt{test}           &             & $r_1$    & Remplit les flags $\textbf{N}$\footnote{\textit{Flag} négatif} et $\textbf{Z}$\footnote{\textit{Flag} zéro} à partir de la valeur de $r_1$. \\
      \texttt{jmp}            &             & $r_1$    & Continue l'exécution du programme à l'adresse $r_1$.                                                                                        \\
      \texttt{jmp}            &             & $\$l$    & Continue l'exécution du programme à l'adresse associée au label $l$.                                                                        \\
      \texttt{jmp}            &             & $i$      & Continue l'exécution du programme à l'adresse $i$ (entier codé sur $16$ bits).                                                              \\
      \texttt{jmp}            &             & $+i$     & Continue l'exécution du programme $i$ adresses (entier codé sur $16$ bits) plus tard.                                                       \\
      \texttt{jmp}            &             & $-i$     & Continue l'exécution du programme $i$ adresses (entier codé sur $16$ bits) avant.                                                           \\
      \texttt{jmp}.\textit{F} &             & $r_1$    & Continue l'exécution du programme à l'adresse $r_1$ lorsque le \textit{flag} $F$ est activé.                                                \\
      \texttt{jmp}.\textit{F} &             & $\$l$    & Continue l'exécution du programme à l'adresse associée au label $l$ lorsque le \textit{flag} $F$ est activé.                                \\
      \texttt{jmp}.\textit{F} &             & $i$      & Continue l'exécution du programme à l'adresse $i$ (entier codé sur $16$ bits) lorsque le \textit{flag} $F$ est activé.                      \\
      \texttt{jmp}.\textit{F} &             & $+i$     & Continue l'exécution du programme $i$ adresses (entier codé sur $16$ bits) plus tard lorsque le \textit{flag} $F$ est activé.               \\
      \texttt{jmp}.\textit{F} &             & $-i$     & Continue l'exécution du programme $i$ adresses (entier codé sur $16$ bits) avant lorsque le \textit{flag} $F$ est activé.                   \\
      \texttt{halt}           &             &          & Saute à l'adresse $2^{32} - 1$. Les valeurs des registres $r_3$ et $r_4$ sont écrasées.                                                     \\
      \bottomrule
\end{tabularx}


\subsubsection*{Instructions pour les Fonctions}

\noindent
\begin{tabularx}{\textwidth}{ccc X}
      \toprule
      Instruction   & Destination & Argument & Description                                                                                                      \\
      \midrule
      \texttt{call} &             & $\$l$    & Appelle la fonction $l$.                                                                                         \\
      \texttt{call} &             & $r_1$    & Appelle la fonction à l'adresse $r_1$ du programme.                                                              \\
      \texttt{ret}  &             &          & Termine la fonction courante et retourne à l'exécution du programme. La valeur du registre $r_{19}$ est écrasée. \\
      \bottomrule
\end{tabularx}

\end{document}
